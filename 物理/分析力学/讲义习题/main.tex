\documentclass[a4paper]{article}
\usepackage{physics}
\usepackage[left=1.5cm,right=1.5cm,top=2.50cm,bottom=1.50cm]{geometry}
\usepackage{ctex}
\title{分析力学部分习题答案}
\author{杨哲涵,郑济东}
\date{\today}
\begin{document}
\maketitle
所有答案编号对应物理系李岩松老师分析力学讲义,正确性只经过有限的检验,可能存在错误,欢迎提出PR修正补充.
\section{第一章习题}
\subsection{约束和自由度}
\paragraph{1}
自行车共有6个自由度,附加1个非完整约束(后轮角速度与车架质心速度).
\paragraph{2}
过程略
\begin{enumerate}
    \item 是非完整约束
    \item 是非完整约束
    \item 是非完整约束
\end{enumerate}
\subsection{虚功原理}
\paragraph{6}
设广义坐标$\theta$为杆与竖直方向的夹角.
\begin{enumerate}
    \item 平衡位置为$\theta=\arcsin\sqrt[3]{\frac{2d}{l}}$,该处$\delta^2W=-\cos\theta(2d\sin^{-3}\theta-\frac{l}{2})(\delta)^2<0$,因此是稳定平衡.
    \item 假设解除墙面约束,设广义坐标为$\alpha,\theta$,有$\delta W=(mg\sin\theta(\frac{l}{2}-a))\delta\theta+(mg\cos\theta-F\sin\theta)\delta\alpha$,可以解得$\sin\theta=\sqrt[3]{\frac{2d}{l}},F=\frac{mg}{\tan\theta}$.
\end{enumerate}
\paragraph{7}
解除上下4根杆的长度约束,令中间杆长度仍然固定.设广义坐标$\phi$为2杆与水平方向的夹角,上方两杆的拉力为$F$,下方两杆的拉力为$T$.
由$\delta W=(P+2T\sin\phi)\delta(2d\tan\phi)+2F\delta\sqrt{(d-2d\tan\phi)^2+(2d)^2}$,令$\phi=0$(对应题目所给位形),可解得$F=\frac{\sqrt{5}}{2}P,T=-P$.因此1杆受压力$P$.
\paragraph{8}
$\theta=0$为平衡位置,有$\delta^2W=\frac{mgl}{2\tan\alpha}(\delta\theta)^2>0$,不是稳定平衡.
\paragraph{11}
设曲面方程为$y=f(x)$,设广义坐标为$r,\theta$,两球的坐标分别为$(x_1,y_1),(x_2,y_2)$
\begin{enumerate}
    \item 曲面方程与平衡位形$(r,\theta=0)$关系为$\frac{m_1}{m_2}=\frac{l-r}{f(x)+\frac{x}{f^\prime(x)}},x^2+f(x)^2=(l-r)^2$
    \item $m_1\sqrt{x^2+y^2}=-m_2y+C$
\end{enumerate}
\paragraph{14}
参考讲义,可得两端固定对虚位移的约束为
$$\int_0^l\frac{\var{y}}{\sqrt{1+(\dv{y}{x})^2}}\dv[2]{y}{x}\dd{s}=0$$
设单位水平间距载荷为$\alpha$,虚功为
$$\var{W}=\int_0^l\frac{\alpha}{\sqrt{1+(\dv{y}{x})^2}}\var{y}\dd{s}$$
由拉格朗日乘子法
$$\int_0^l(\frac{\alpha}{\sqrt{1+(\dv{y}{x})^2}}-\frac{\lambda}{\sqrt{1+(\dv{y}{x})^2}}\dv[2]{y}{x})\var{y}\dd{s}=0$$
解得$y=\frac{\alpha}{2\lambda}x^2+C_1x+C_2$,形状为抛物线.
\section{第二章习题}
\subsection{达朗贝尔原理}
\paragraph{2}
设杆长为$2l$,广义坐标为$\alpha$,由达朗贝尔原理
$$\var{W}=mgl\sin\alpha\var{\alpha}-m\ddot{\va*{r}}\var{\va*{r}}-I\ddot{\theta}\var{\theta}$$
解得$\ddot{\alpha}=\frac{3g}{4l}\sin\alpha$,根据牛顿力学可算出墙面支持力为$\frac{3}{4}mg\sin\alpha\cos\alpha$,地面支持力$(1-\frac{3}{4}\sin^2\alpha)mg$.
\subsection{拉格朗日力学}
\paragraph{5}
广义势函数$V=\va{\sigma}\dotproduct(\va{r}\crossproduct\va{p})=m\va{\sigma}\dotproduct(\va{r}\crossproduct\va{r})$
\begin{align*}
    Q_x&=\dv{}{t}\left(\pdv{V}{\dot{x}}\right)-\pdv{V}{x}=2m(\sigma_y\dot{z}-\sigma_z\dot{y}) \\
    Q_y&=\dv{}{t}\left(\pdv{V}{\dot{y}}\right)-\pdv{V}{y}=2m(\sigma_z\dot{x}-\sigma_x\dot{z}) \\
    Q_z&=\dv{}{t}\left(\pdv{V}{\dot{z}}\right)-\pdv{V}{z}=2m(\sigma_x\dot{y}-\sigma_y\dot{x})
\end{align*}
即$\va*{Q}=2\va*{\sigma}\crossproduct\va*{p}$,这在球坐标下也成立.
\paragraph{7}
解拉格朗日方程$\dv{t}(\pdv{L}{\dot{q_k}})-\pdv{L}{q_k}=0$有
\begin{align*}
    & a\ddot{x}+b\ddot{y}+\frac{k}{m}(ax+by)=0 \\
    & b\ddot{x}+c\ddot{y}+\frac{k}{m}(bx+cy)=0
\end{align*}
\begin{enumerate}
    \item 系统运动方程为
    $$\mqty(A_1\sin(\omega t+\phi_1) \\ A_2\sin(\omega t+\phi_2))=\mqty(a&b\\b&c)\mqty(x\\y)\qq{其中}\omega=\frac{k}{m}$$
\end{enumerate}

\paragraph{8}
\section{第三章习题}
\subsection{非完整约束}
\subsection{拉格朗日方程的应用}
\paragraph{2}
\paragraph{5}
\paragraph{6}
\paragraph{7}
\subsection{耗散函数}
\paragraph{9}
\section{第四章习题}
\subsection{变分极值问题}
\paragraph{1}
直线应满足$\rho\ddot{\rho}=2\dot{\rho}^2+\rho^2$
\paragraph{2}
\paragraph{5}
\subsection{最小作用量原理}
\paragraph{10}
\subsection{诺特定理}
\paragraph{11}
\subsection{罗斯函数与罗斯方程}
\paragraph{15}
\paragraph{18}
\section{第五章习题}
\subsection{有心力}
\paragraph{2}
\paragraph{3}
\paragraph{5}
\paragraph{7}
\paragraph{9}
\paragraph{10}
\end{document}
