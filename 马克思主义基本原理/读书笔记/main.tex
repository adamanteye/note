\documentclass[a4paper]{article}
\usepackage[left=1.5cm,right=1.5cm]{geometry}
\usepackage{ctex}
\usepackage[thehwcnt=\hspace{-2pt}]{iidef}
\usepackage[ngerman]{babel} % German language support
% Suppport for German quotation marks https://tex.stackexchange.com/questions/150945/automatic-german-quotation-marks
\usepackage{csquotes}
\MakeOuterQuote{"}
\usepackage{setspace}
\addtolength{\parskip}{0.4em}
\usepackage{hyperref}
\usepackage{cleveref}
\hypersetup{colorlinks = true, linkcolor = red}
\thecourseinstitute{清华大学}
\thecoursename{马克思主义基本原理}
\theterm{2023年秋季学期}
\hwname{读书笔记}
\begin{document}
\courseheader
\name{杨哲涵 2022011105}
我在这学期内,主要通过马克思主义文库在线阅读相关著作,较为认真地读过资本论的第一卷,也根据自己的兴趣选择性地阅读了若干其他内容(包括维基百科的相关页面)。作为本学期总结的读书笔记的字数为6850字左右(由编译前的\LaTeX 统计)。
\section{《资本论》第一卷}
我参考的中文版是\href{https://www.marxists.org/chinese/marx-engels/23/index.htm}{马克思:《资本论》第一卷,中文马克思主义文库}。德文版是\emph{\href{http://www.mlwerke.de/me/me23/me23_000.htm}{Das Kapital. Band I, Marxists’ Internet Archive - Deutschsprachiger Teil}}.
\subsection{商品和货币}
\emph{“每一种有用物,如铁、纸等等,都可以从质和量两个角度来考察。每一种这样的物都是许多属性的总和,因此可以在不同的方面有用。发现这些不同的方面,从而发现物的多种使用方式,是历史的事情。为有用物的量找到社会尺度,也是这样。商品尺度之所以不同,部分是由于被计量的物的性质不同,部分是由于约定俗成。物的有用性使物成为使用价值。但这种有用性不是悬在空中的。它决定于商品体的属性,离开了商品体就不存在。因此,商品体本身,例如铁、小麦、金钢石等等,就是使用价值,或财物。商品体的这种性质,同人取得它的使用属性所耗费的劳动的多少没有关系。在考察使用价值时,总是以它们有一定的量为前提,如几打表,几码布,几吨铁等等。商品的使用价值为商品学这门学科提供材料。使用价值只是在使用或消费中得到实现。不论财富的社会形式如何,使用价值总是构成财富的物质内容。在我们所要考察的社会形式中,使用价值同时又是交换价值的物质承担者。”}

这是资本论的开篇,有趣的是马克思在德文中的用词是\emph{"Gebrauchswert"},德语中的"brauchen"与"nutzen"含义是不一样的。读到这里时,感觉\emph{"Gebrauchswert"}直译为“需求价值”似乎也是合理的,更直接地反映出商品的使用价值是因为人们的需求而产生的,并且这也指明了,使用价值只在使用或消费中得到实现,与人们取得它的使用属性所耗费的劳动的多少没有关系。

\emph{“交换价值首先表现为一种使用价值同另一种使用价值相交换的量的关系或比例,这个比例随着时间和地点的不同而不断改变。因此,交换价值好象是一种偶然的、纯粹相对的东西,也就是说,商品固有的、内在的交换价值似乎是一个形容语的矛盾。”}

交换价值原文为\emph{"Tauschwert"},这里采取了直译。

\emph{“作为使用价值,商品首先有质的差别;作为交换价值,商品只能有量的差别,因而不包含任何一个使用价值的原子……现在我们来考察劳动产品剩下来的东西。它们剩下的只是同一的幽灵般的对象性,只是无差别的人类劳动的单纯凝结,即不管以哪种形式进行的人类劳动力耗费的单纯凝结。这些物现在只是表示,在它们的生产上耗费了人类劳动力,积累了人类劳动。这些物,作为它们共有的这个社会实体的结晶,就是价值——商品价值。”}

\emph{“形成价值实体的劳动是相同的人类劳动,是同一的人类劳动力的耗费。体现在商品世界全部价值中的社会的全部劳动力,在这里是当作一个同一的人类劳动力,虽然它是由无数单个劳动力构成的。每一个这种单个劳动力,同别一个劳动力一样,都是同一的人类劳动力,只要它具有社会平均劳动力的性质,起着这种社会平均劳动力的作用,从而在商品的生产上只使用平均必要劳动时间或社会必要劳动时间。社会必要劳动时间是在现有的社会正常的生产条件下,在社会平均的劳动熟练程度和劳动强度下制造某种使用价值所需要的劳动时间。”}

随后,马克思还谈到了商品的二重性,即价值和使用价值。使用价值是商品的自然属性,具有不可比较性。价值是一般人类劳动的凝结,是商品的社会属性,它构成商品交换的基础。总结马克思的劳动价值理论,即商品具有二重性,商品价值量由生产这种商品的社会必要劳动时间决定。
\subsection{货币转化为资本}
\emph{“例如,用100镑买的棉花卖100镑+10镑,即110镑。因此,这个过程的完整形式是$G—W—G^\prime$。其中的$G^\prime=G+\Delta G$,即等于原预付货币额加上一个增殖额。我把这个增殖额或超过原价值的余额叫做剩余价值。可见,原预付价值不仅在流通中保存下来,而且在流通中改变了自己的价值量,加上了一个剩余价值,或者说增殖了。正是这种运动使价值转化为资本。”}

德文中的\emph{"Mehrwert"}意思更为直接,是“多余的价值”或“附加的价值”等等。当然,译文在这里用“剩余价值”,是强调其来自于资本家从工人身上的剥削。

\emph{“因此,那些试图把商品流通说成是剩余价值的源泉的人,其实大多是弄混了……无论怎样颠来倒去,结果都是一样。如果是等价物交换,不产生剩余价值;如果是非等价物交换,也不产生剩余价值……剩余价值不能从流通中产生;因此,在剩余价值的形成上,必然有某种在流通中看不到的情况发生在流通的背后。”}

马克思证明了流通并不会让商品直接增值,剩余价值的产生另有秘密,他即将在剩余价值理论中揭示这个秘密。

\emph{“作为这一运动的有意识的承担者,货币所有者变成了资本家。他这个人,或不如说他的钱袋,是货币的出发点和复归点。这种流通的客观内容——价值增殖——是他的主观目的;只有在越来越多地占有抽象财富成为他的活动的唯一动机时,他才作为资本家或作为人格化的、有意志和意识的资本执行职能。因此,绝不能把使用价值看作资本家的直接目的。他的目的也不是取得一次利润,而只是谋取利润的无休止的运动。”}

多么精辟的论述!资本家自己的每次买卖,都遵循$G-W-G^\prime$的资本流通公式。商品的生产是否有利可图,是资本家唯一考虑的问题。而商品的生产是否过剩,是否会造成社会危机,资本家没有兴趣也没有能力去考虑。
\subsection{绝对剩余价值的生产}
\emph{“作为买者,资本家对每一种商品——棉花、纱锭和劳动力——都按其价值支付。然后他做了任何别的商品购买者所做的事情。他消费它们的使用价值。劳动力的消费过程(同时是商品的生产过程)提供的产品是20磅棉纱,价值30先令。资本家在购买商品以后,现在又回到市场上来出售商品。他卖棉纱是1先令6便士一磅,既不比它的价值贵,也不比它的价值贱。然而他从流通中取得的货币比原先投入流通的货币多3先令。他的货币转化为资本的这整个过程,既在流通领域中进行,又不在流通领域中进行。它是以流通为媒介,因为它以在商品市场上购买劳动力为条件。它不在流通中进行,因为流通只是为价值增殖过程作准备,而这个过程是在生产领域中进行的。”}

马原课上讲到,“剩余价值的完成,既不在流通中进行,又必须在流通中进行”。这就阐释了剩余价值形成的秘密,资本家购买并消费了劳动力后,需要将商品再投入到流通中,才能实现剩余价值的增值,尽管消费劳动力的过程使得价值得到了增值,但没有资本家“惊险的一跃”,剩余价值仍然无法得到实现。
\subsection{相对剩余价值的生产}
\emph{“劳动生产力的提高,在这里一般是指劳动过程中的这样一种变化,这种变化能缩短生产某种商品的社会必需的劳动时间,从而使较小量的劳动获得生产较大量使用价值的能力。在研究我们上面考察的那种形式的剩余价值的生产时,我们曾假定生产方式是既定的。而现在,对于由必要劳动变成剩余劳动而生产剩余价值来说,资本只是占有历史上遗留下来的或者说现存形态的劳动过程,并且只延长它的持续时间,就绝对不够了。必须变革劳动过程的技术条件和社会条件,从而变革生产方式本身,以提高劳动生产力,通过提高劳动生产力来降低劳动力的价值,从而缩短再生产劳动力价值所必要的工作日部分。我把通过延长工作日而生产的剩余价值,叫做绝对剩余价值;相反,我把通过缩短必要劳动时间、相应地改变工作日的两个组成部分的量的比例而生产的剩余价值,叫做相对剩余价值。”}

这里说的工作日的两个组成部分分别是必要劳动与剩余劳动。马克思这里指出了,资本主义生产下,发展劳动生产力的目的,不是为了让工人工作时间更短更轻松,而是为了缩短生产商品所必要的劳动时间,工人为必须为自己劳动的时间缩短了,那么工人无偿为资本家劳动的时间就增长了。我们看到,资本家继续赚取剩余价值的主要来源是相对剩余价值。

\emph{“整个社会内的分工,不论是否以商品交换为媒介,是各种社会经济形态所共有的,而工场手工业分工却完全是资本主义生产方式的独特创造……要得到进一步分工的利益,就必须进一步增加工人人数,而且只能按倍数来增加。但是随着资本的可变部分的增加,资本的不变部分也必须增加,建筑物、炉子等共同生产条件的规模要扩大,原料尤其要增加,而且要比工人人数快得多地增加。由于分工,劳动生产力提高了,一定劳动量在一定时间内消耗的原料数量也就按比例增大。因此,单个资本家手中的资本最低限额越来越增大,或者说,社会的生活资料和生产资料越来越多地转化为资本,这是由工场手工业的技术性质产生的一个规律。”}

马克思在劳动分工的问题上,继承了亚当·斯密的观点,他最闻名的例子莫过于制针业的分工。马克思除了像斯密一样,赞扬劳动分工对生产力的巨大提升、批判单调工作对工人的伤害外,也指出劳动分工使得生产过程越来越复杂,规模越来越庞大,从而促进了社会的财富以资本的形式集中。

\emph{“机器生产相对剩余价值,不仅由于它直接地使劳动力贬值,使劳动力再生产所必需的商品便宜,从而间接地使劳动力便宜,而且还由于它在最初偶而被采用时,会把机器所有主使用的劳动变为高效率的劳动,把机器产品的社会价值提高到它的个别价值以上,从而使资本家能够用日产品中较小的价值部分来补偿劳动力的日价值……随着机器在同一生产部门内普遍应用,机器产品的社会价值就降低到它的个别价值的水平”}

今天大语言模型非常流行,这段话现在来看一点也不过时,率先使用人工智能工具的能够提升个别的生产力,但在普遍应用后,人工智能协助生产的产品的社会价值就降低了。
\section{剩余价值理论:《资本论》第四卷}
我参考的中文版是\href{https://www.marxists.org/chinese/marx-engels/26-1/006.htm}{马克思:剩余价值理论,中文马克思主义文库}。德文版是\emph{\href{https://www.marxists.org/deutsch/archiv/marx-engels/1863/tumw/standard/index.htm}{Theorien über den Mehrwert, Marxists’ Internet Archive - Deutschsprachiger Teil}}.
\subsection{关于生产劳动和非生产劳动的理论}
马克思在这一章讨论了亚·斯密的诸多观点,例如生产劳动与非生产劳动的二重性,批判了众多庸俗资产阶级经济学家的观点。

\emph{“但是罗西断言,就连专门用来使主人摆阔、满足主人虚荣心的那些家仆的“劳动”,也“不是非生产劳动”。为什么呢?因为它生产某种东西:满足虚荣心,使主人能够吹嘘、摆阔(同上,第277页)。这里我们又看到了那种胡说八道,好象每种服务都生产某种东西:妓女生产淫欲,杀人犯生产杀人行为等等。而且,据说斯密说过,这些污秽的东西每一种都有自己的价值。就差说这些“服务”是无酬的了。问题并不在这里。但是,即使这些服务是无酬的,它们也不会使财富(物质财富)增加一文钱。}

\emph{然后又是一段美文学式的胡言乱语:}

\emph{『有人硬说,歌手唱完歌,不给我们留下什么东西。不,他留下回忆!〈妙极了!〉你喝完香槟酒留下了什么呢?……消费是否紧紧跟随生产,消费进行得快还是慢,固然会使经济结果有所不同,但消费这个事实本身无论怎样也不会使产品丧失财富的性质。某些非物质产品比某些物质产品存在更长久。一座宫殿会长期存在,但《伊利亚特》是更长久的享受来源。”(第277—278页)』}

\emph{多么荒唐!}

\emph{从这里罗西所理解的财富的意义,即从使用价值的意义来说,情况甚至是这样的:只有消费才使产品成为财富,而不管这种消费是快还是慢(消费的快慢决定于消费本身的性质和消费品的性质)。使用价值只对消费有意义,而且对消费来说,使用价值的存在,只是作为一种消费品的存在,只是使用价值在消费中的存在。喝香槟酒虽然生产“头昏”,但不是生产的消费,同样,听音乐虽然留下“回忆”,但也不是生产的消费。如果音乐很好,听者也懂音乐,那末消费音乐就比消费香槟酒高尚,虽然香槟酒的生产是“生产劳动”,而音乐的生产是非生产劳动。”}

读到这一段时不得不感慨马克思犀利幽默的文风,他在这里批判了佩·罗西关于非生产劳动者“节约劳动”的庸俗见解。佩·罗西认为,非生产劳动者间接地促进了物质生产,为了制造一顶帽子,街上巡逻的宪兵、坐在法庭的法官、关押凡人的狱警、守卫国境的军队都促进生产。仔细考察这种说法的合理性,法官没有农民便吃不上饭,那么农民促进了司法的生产吗?如此就看出这种说法的荒谬之处了。

佩·罗西自己还重复了加尔涅的意见,认为非生产劳动如果节约了产业工人、资本家等等的时间,让他们有更多的时间从事生产劳动,那么“非生产劳动就是生产的”。马克思驳斥这种观点时,采用分工的看法。每个人从事生产劳动外,都要完成做饭,打扫屋子,剪发,看病等任务,其中一些任务还有消费的功能。但是社会分工使得非生产劳动者可以将上面这些非生产劳动变成自己的专门职能,而其他人可以将生产劳动作为自己的专业职能。因此在这个问题上,不应故意混淆两种劳动的界限。

\emph{“这是还具有革命性的资产阶级说的话,那时它还没有把整个社会、国家等等置于自己支配之下。所有这些卓越的历来受人尊敬的职业——君主、法官、军官、教士等等,所有由这些职业产生的各个旧的意识形态阶层,所有属于这些阶层的学者、学士、教士……在经济学上被放在与他们自己的、由资产阶级以及有闲财富的代表(土地贵族和有闲资本家)豢养的大批仆从和丑角同样的地位。他们不过是社会的仆人,就象别人是他们的仆人一样。他们靠别人劳动的产品生活。因此,他们的人数必须减到必不可少的最低限度。国家、教会等等,只有在它们是管理和处理生产的资产者的共同利益的委员会这个情况下,才是正当的;这些机构的费用必须缩减到必要的最低限度,因为这些费用本身属于生产上的非生产费用。这种观点具有历史的意义,一方面,它同古代的见解形成尖锐的对立,在古代,物质生产劳动带有奴隶制的烙印,这种劳动被看作仅仅是有闲的市民的立足基石;另一方面,它又同由于中世纪瓦解而产生的专制君主国或贵族君主立宪国的见解形成尖锐的对立…… 相反,一旦资产阶级占领了地盘,一方面自己掌握国家,一方面又同以前掌握国家的人妥协;一旦资产阶级把意识形态阶层看作自己的亲骨肉,到处按照自己的本性把他们改造成为自己的伙计;一旦资产阶级自己不再作为生产劳动的代表来同这些人对立,而真正的生产工人起来反对资产阶级,并且同样说它是靠别人劳动生活的;一旦资产阶级有了足够的教养,不是一心一意从事生产,而是也想从事“有教养的”消费;一旦连精神劳动本身也愈来愈为资产阶级服务,为资本主义生产服务;——一旦发生了这些情况,事情就反过来了。这时资产阶级从自己的立场出发,力求“在经济学上”证明它从前批判过的东西是合理的。加尔涅等人就是资产阶级在这方面的代言人和良心安慰者。此外,这些经济学家(他们本人就是教士、教授等等)也热衷于证明自己“在生产上的”有用性,“在经济学上”证明自己的薪金的合理性。”}

这段议论不禁让人想到共产党宣言中的一段话:\emph{“资产阶级抹去了一切向来受人尊崇和令人敬畏的职业的神圣光环。它把医生、律师、教士、诗人和学者变成了它出钱招雇的雇佣劳动者”}。这正说明了资本主义生产方式取得的巨大成功,一切不事生产,有碍生产的旧阶层都为资产阶级所不齿,资产阶级按照自己的意愿改造了世界,知识分子在资产阶级眼中也不过是高级的雇用劳动者而已。马克思也英明地预见到了资产阶级的庸俗化,当资产阶级不再一心一意从事生产,也想迈入“上流社会”时,资产阶级又开始对旧世界的封建残余进行各种妥协,庸俗地同流合污起来。当今中国社会的郭继承之流就是一例,从事俗不可耐的“非生产劳动”,作为思想封建残余的代言人,被中国乡土资产阶级所包养,安慰着土老板与精神土老板贫瘠的灵魂。
\section{哥达纲领批判}
我参考的版本是\href{https://www.marxists.org/chinese/marx/marxist.org-chinese-marx-1875-4.htm}{马克思:哥达纲领批判,中文马克思主义文库}。

查阅了一番背景资料后,得知马克思的写作背景是德国社会主义工人党合并成立之时,德国社会主义工人党就是德国社会民主党(SPD)的前身,如今是德国红绿灯政府中的一员。

\emph{『工人阶级为了本身的解放,首先是在现代民族国家的范围内进行活动,同时意识到,它的为一切文明国家的工人所共有的那种意图必然导致的结果,将是各民族的国际的兄弟联合。』}

\emph{“同‘共产党宣言’和先前的一切社会主义相反,拉萨尔以最狭隘的民族观点来对待工人运动。有人竟在这方面追随他,而且这是在国际的活动以后……这个纲领的国际主义,比那个自由贸易派的国际主义还差得难以估量。自由贸易派也说,它的意图所导致的结果是“各民族的国际的兄弟联合”。但是它还做一些事使贸易成为国际性的,而决不满足于一切民族各自在本国内从事贸易的意识。”}

马克思对这条纲领的批判,在于纲领放弃了国际主义,转而大谈“现代民族国家”,这样的话听起来简直是右翼民族主义的味道,至少一点也不像工人阶级的政党。对于国际运动的职责,纲领也只说“那种意图必然导致的结果”,不愿亲自辛劳,些微的使命感也难以见到。

\emph{『德国工人党从这些原则出发,力求用一切合法手段来争取自由国家——和——社会主义社会,废除工资制度连同铁的工资现律——和——任何形式的剥削,消除一切社会的和政治的不平等。』}

\emph{“这样,德国工人党将来就不得不相信拉萨尔的“铁的工资规律”了!为了不让它埋没掉,竟胡说什么“废除工资制度(应当说:雇佣劳动制度)连同铁的工资规律”。如果我废除了雇佣劳动,那末我当然也废除了它的规律,不管这些规律是“铁的”还是海绵的。但是拉萨尔反对雇佣劳动的斗争几乎只是绕着这个所谓规律兜圈子。所以,为了证明拉萨尔派已经获得胜利,“工资制度连同铁的工资规律”都应当被废除掉,而不是不连同后者。大家知道,在“铁的工资规律”中,除了从歌德的“永恒的、铁的、伟大的规律”中抄来的“铁的”这个词以外,没有一样东西是拉萨尔的……这正像奴隶们最终发现了自己处于奴隶地位的秘密而举行起义时,其中有一个为陈旧观点所束缚的奴隶竟要在起义的纲领上写道:奴隶制度必须废除,因为在奴隶制度下,奴隶的给养最大限度不能超过一定的、非常低的标准!”}

马克思读到纲领中的这一条时应当是气愤之极,以至于又花了四五行的篇幅再谈一遍雇佣劳动导致雇佣工人白白为资本家创作剩余价值的事实,愤怒于工人党内竟然没有人指出这个轻率、严重的错误,任由其滑过去了。这样的外行话简直难以让别人相信他们是一个工人党。当然,气愤之余,马克思又不吝笔墨地嘲讽了那个煞有介事、从歌德那里抄来的中二用词。

读完哥达纲领批判后,我的收获之一是看到了马克思怎么分析政治纲领文件,细致入微地阅读每个条文,一眼识破其中煞有介事、故弄玄虚的部分,不仅从写在纸上的文字分析其立场,还从没写在纸上的地方分析起草者无意或故意模糊化的思想。这样的分析方法,对于了解一个政治组织的立场、动机、思想、目的都有很大的帮助。

谈一句题外话,德国社民党早已不再把马克思主义当作指导思想,左翼党(Die Linke)与德国的共产党(Deutsche Kommunistische Partei)吸纳了社民党内部的马克思主义派别。德国共运的命运令人唏嘘。
\section{青年在选择职业时的考虑}
这篇文章的思想价值并不能比拟其他著作,但对于在校大学生而言也有触动之处,我参考的中文版是\href{https://www.marxists.org/chinese/marx/marxist.org-chinese-marx-1835-8.htm}{马克思:青年在选择职业时的考虑,中文马克思主义文库},德文版是\href{http://www.zeno.org/Philosophie/M/Marx,+Karl/Betrachtungen+eines+J%C3%BCnglings+bei+der+Wahl+seines+Berufes}{Karl Marx: Betrachtung eines Jünglings bei der Wahl eines Berufes, zeno.org}.

\emph{"Wenn wir den Stand gewählt, in dem wir am meisten für die Menschheit wirken können, dann können uns Lasten nicht niederbeugen, weil sie nur Opfer für alle sind; dann genießen wir keine arme, eingeschränkte, egoistische Freude, sondern unser Glück gehört Millionen, unsere Taten leben still, aber ewig wirkend fort, und unsere Asche wird benetzt von der glühenden Träne edler Menschen."}

这一部分不涉及马克思的伟大的分析方法与思想,但独特之处在于,马克思其他的文字当中,辛辣、嘲讽、幽默的风格都很常见,但是如此抒情和较为直白,也易于理解的段落比较少见。我因为初学者外加受到感动的缘故,把它当作德语写作的例子摘录下来。
\end{document}