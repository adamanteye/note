\documentclass[a4paper]{article}
\usepackage[left=1.5cm,right=1.5cm]{geometry}
\usepackage{ctex}
\usepackage[thehwcnt = 1]{iidef}
\thecourseinstitute{清华大学}
\thecoursename{计算机网络原理}
\theterm{2023年秋季学期}
\hwname{作业}
\begin{document}
\courseheader
\name{杨哲涵}
以下是我阅读\textit{The Design Philosophy of the DARPA Internet Protocols}后自己总结的要点与思考:
\paragraph{成功实现的核心目标}
这一新体系架构的核心目标是连通已有的不同网络,为此采用了分组交换,网络层网关(存储转发)技术.TCP/IP在这一方面获得了巨大的成功.
\paragraph{目标优先级的影响}
作者提到,需求优先级次序对实际设计有巨大的影响.军方对生存性的要求导致了一系列设计选择.

例如,衡量资源占用的需求被牺牲掉了,最终没有得到很好的实现,不得不说是一种缺憾.

但是,最优先的生存性(在部分节点的损失下运作)带来了很多非常棒的概念和设计:「命运共享原则」拒绝通过中间节点维护状态,而是通过端到端的连接提供了诸多网络层的保证(只要能够连接,那么状态就是同步的),简化了协议的实现;对物理网络的抽象很保守(仅要求提供点对点的连通路径),使得对不同网络的适应性很高.

还有,DARPA关于分布式配置管理的需求实际上为商业互联网和个人计算机的发展提供了巨大便利,即便它也引入了一些风险.假如Internet没有如此便利的分布式特性,将很难支撑如今规模的互联网正常发展.
\paragraph{数据报与服务}
数据报(datagram)是一个相当成功的概念.它几乎是对网络的最小抽象,因而也相当底层,足够基础.TCP/IP在传输层需要提供可定制的协议,这些协议都是在网络层的数据报之上构建的.这也许可以称为「最小设计原则」.对数据报,我们只要求最佳性能.可靠性可以委托更上层的协议(TCP)来提供.TCP与后来出现的HTTP的关系正类似于此.
\paragraph{设计缺憾}
论文作者也谈到了TCP/IP的一些缺憾.例如没有考虑资源占用测量的需求(当然实现起来也有麻烦);由于灵活性与「最小设计」,对实际网络(作者称为Realization)没有规范,这一问题被留给了工程师们;基于相同的原因,不能针对特定网络进行调整以获得最优性能;没有性能衡量工具(这也因为如何定义"性能"存在困难);报头太长,浪费传输资源等等.
\paragraph{感想与思考}
感慨TCP/IP的成功,体会设计哲学之余,它也的确存在着缺憾.

例如,在OSI参考模型中,层间的接口不变,采用的协议应当可以被方便地升级替换,TCP/IP却没能满足这一思想.

此外,TCP/IP发明之时还没有网络安全的概念.但很明显的是,下一代网络可能需要在较底层(网络层或者传输层)提供不可抓取,不可伪造的保证.否则,更上层的应用反复实现同样功能未免是一种浪费,网络监视,网络污染仍然难以避免.

最后,探索更加定制化的,更容易实现滚动升级与扩展的体系架构设计也是一种对TCP/IP设计哲学的创新,特别是将近40年网络发展的经验融入到设计哲学当中.
\end{document}
