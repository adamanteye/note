\documentclass[a4paper]{article}
\usepackage[left=1.5cm,right=1.5cm]{geometry}
\usepackage{ctex}
\usepackage[thehwcnt = 2]{iidef}
\usepackage{fontspec}
\usepackage{listings}
\usepackage{xcolor}

% 设置 Noto Sans Mono 字体
\setmonofont{Noto Sans Mono}[Scale=MatchLowercase]

% 设置代码样式
\lstset{
    basicstyle          =   \ttfamily,          % 基本代码风格
    keywordstyle        =   \bfseries,          % 关键字风格
    commentstyle        =   \itshape,           % 注释的风格,斜体
    stringstyle         =   \ttfamily,          % 字符串风格
    flexiblecolumns,                          % 别问为什么,加上这个
    numbers             =   left,               % 行号的位置在左边
    showspaces          =   false,              % 是否显示空格
    numberstyle         =   \footnotesize\ttfamily,    % 行号的样式,小五号,tt等宽字体
    showstringspaces    =   false,
    captionpos          =   t,                  % 这段代码的名字所呈现的位置,t指的是top上面
    frame               =   lrtb,               % 显示边框
}

\lstdefinestyle{Python}{
    language        =   Python,
    basicstyle      =   \footnotesize\ttfamily,
    numberstyle     =   \footnotesize\ttfamily,
    keywordstyle    =   \color{blue},
    keywordstyle    =   [2] \color{teal},
    stringstyle     =   \color{magenta},
    commentstyle    =   \color{red}\itshape,
    breaklines      =   true,                 % 自动换行
    columns         =   fixed,                % 字间距固定
}
\thecourseinstitute{清华大学}
\thecoursename{决策方法论}
\theterm{2023年秋季学期}
\hwname{作业}
\begin{document}
\courseheader
\name{杨哲涵}
~
\paragraph{例题分析的深化}
如果现在希望在五天的救灾活动之后,仍然能保持不少于350辆的完好汽车.
那么可以修改代价函数,通过一个因子$\lambda$严厉惩罚汽车数量少于350辆的情况.

更新后的贝尔曼方程如下

$$J_k(S_k)=\max_{x_k\in A_k(S_k)}(5S_k+3x_k-\lambda\max(0,350-S_{k+1})+J_{k+1}(0.9S_k-0.2x_k))$$

我写了Python代码来模拟并计算最佳策略,$\lambda$值被设置为10000(只要够大就行).经过计算,最终结果如表所示:

\begin{table}[h]
    \centering
    \caption{动态规划执行结果}
    \begin{tabular}{|c|c|c|c|c|c|c|}
        \hline
        \textbf{Days}  & 1    & 2   & 3   & 4   & 5   & 6   \\ \hline
        \textbf{$S_k$} & 1000 & 900 & 810 & 714 & 500 & 350 \\ \hline
        \textbf{$x_k$} & 0    & 0   & 75  & 713 & 500 &     \\ \hline
    \end{tabular}
\end{table}

成功运送的物资为23483.

此外,如果取$\lambda=0$,那么情形将退化到原始情况,如果取$\lambda\max(0,350-S_k)$中的350为其他数值,则可计算其他希望的情况,代码的通用性很好.

代码附后:

\newpage
\begin{lstlisting}
    import numpy as np
    def optimal_strategy(lambda_val):
        J = np.zeros((7, 1001))
        strategy = np.zeros((7, 1001))
        for k in range(6, 0, -1):
            for Sk in range(1000, -1, -1):
                max_value = -1
                best_xk = 0
                for xk in range(0, Sk + 1):
                    if k == 6:
                        J[k][Sk] = 0
                        continue
                    reward = 5 * Sk + 3 * xk - lambda_val * max(0, 350 - int(0.9 * Sk - 0.2 * xk))
                    next_state = int(0.9 * Sk - 0.2 * xk)
                    value = reward + J[k + 1][next_state]
                    if value > max_value:
                        max_value = value
                        best_xk = xk
                J[k][Sk] = max_value
                strategy[k][Sk] = best_xk
        return J, strategy
    lambda_val = 10000 # 0 表示允许少于 350
    J_values, optimal_strategies = optimal_strategy(lambda_val)
    print("Optimal total material transported:", J_values[1][1000])
    s1=1000
    s2=int(s1*0.9-optimal_strategies[1][s1]*0.2)
    s3=int(s2*0.9-optimal_strategies[2][s2]*0.2)
    s4=int(s3*0.9-optimal_strategies[3][s3]*0.2)
    s5=int(s4*0.9-optimal_strategies[4][s4]*0.2)
    s6=int(s5*0.9-optimal_strategies[5][s5]*0.2)
    print(s1,s2,s3,s4,s5,s6)
    print(optimal_strategies[1][s1],optimal_strategies[2][s2],optimal_strategies[3][s3],
optimal_strategies[4][s4],optimal_strategies[5][s5])
\end{lstlisting}

\end{document}
