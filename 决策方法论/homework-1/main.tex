\documentclass[a4paper]{article}
\usepackage[left=1.5cm,right=1.5cm]{geometry}
\usepackage{ctex}
\usepackage[thehwcnt = 1]{iidef}
\thecourseinstitute{清华大学}
\thecoursename{决策方法论}
\theterm{2023年秋季学期}
\hwname{作业}
\begin{document}
\courseheader
\name{杨哲涵}
~
\paragraph{举例说明四种马尔科夫过程}
以下四种过程都有马尔科夫特性(状态变量无后效性)
\begin{itemize}
    \item 求最短路的Dijkstra算法.Dijkstra算法具有马尔科夫特性.在算法执行的每一步,它只考虑从起始节点到当前节点的最短路径,并选择下一个节点,这个选择只依赖于当前节点的状态,不依赖于之前的节点.
    \item 棋类,纸牌游戏.这些游戏通常也具有马尔科夫特性.在游戏中的每一步,玩家的决策只依赖于当前棋盘或牌局的状态,以及他们的策略.之前的游戏状态并不直接影响将来的决策.这是因为在许多棋类和纸牌游戏中,玩家只关注当前局面和可能的下一步,而不需要考虑整个游戏历史.
    \item 随机行走是一个具有马尔科夫特性的过程.在随机行走中,一个对象(例如粒子或者游走者)在每一步中随机选择一个方向前进.这个选择只取决于当前位置,与之前的步骤无关.这种特性使得随机行走可以建模许多随机过程,包括布朗运动和马尔科夫链.
    \item 一些气象学模型使用马尔科夫特性来描述天气的变化.在这些模型中,天气状态被视为一个离散的状态空间,如晴天、多云、雨天等.每一天的天气状态只依赖于前一天的状态,而不受之前的天气状态的影响.这使得天气状态的转变可以用马尔科夫链来建模,其中每个状态代表一种天气条件,而状态之间的转移概率只依赖于前一天的天气.
\end{itemize}
\paragraph{机器最优维修策略过程}
\begin{itemize}
    \item 阶段$T$: 当前所处的天数($t=0,1,2,\cdots$)
    \item 状态$S$: 系统有两个状态,正常运行($i=1$)与发生故障($i=2$)
    \item 决策$a_t$: 所有可能的决策为正常生产($a_1$),全面修理($a_2$),简单修理($a_3$)
    \item 状态转移概率$P_a(x,y)$: $P_{a_1}(1,1)=1,P_{a_2}(2,1)=0.6,P_{a_2}(2,2)=0.4,P_{a_3}(2,1)=0.4,P_{a_3}(2,2)=0.6$
    \item 成本$g_a(x)$: 为了成本最小,取收益为负值,支出为正值. $g_{a_1}(1)=-10,g_{a_2}(2)=5,g_{a_3}(2)=2$
\end{itemize}
\end{document}