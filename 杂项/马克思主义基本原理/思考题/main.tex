\documentclass[a4paper]{article}
\usepackage[left=1.5cm,right=1.5cm]{geometry}
\usepackage{ctex}
\usepackage[thehwcnt=\hspace{-2pt}]{iidef}
\usepackage{hyperref}
\usepackage{cleveref}
\hypersetup{colorlinks = true, linkcolor = red}
\thecourseinstitute{清华大学}
\thecoursename{马克思主义基本原理}
\theterm{2023年秋季学期}
\hwname{思考题}
\begin{document}
\courseheader
\name{杨哲涵}
~
\paragraph{题目}
马克思在《资本论》第一卷第13章中科学地预言:“最后,大工业领域内生产力的极度提高,以及随之而来的所有其他生产部门对劳动力的剥削在内涵和外延两方面的加强,使工人阶级中越来越大的部分有可能被用于非生产劳动,特别是使旧式家庭奴隶在“仆役阶级”(如仆人、使女、侍从等等)的名称下越来越大规模地被再生产出来。

现代社会中越来越多的劳动者并不直接从事生产性劳动,而是从事非生产性劳动的工作,如企业管理人员、警察、律师、医生、会计、保姆、销售、店员、中介、公务员等第三产业的职业。那么这些职业的劳动并不直接形成价值,因此也就也不创造剩余价值,那么他们属于剥削阶级吗?

因此,有些人指责知识分子从事的不是生产劳动,属于剥削者,从而提出想要修改马克思的劳动价值论,主张知识也创造价值(真实目的在于,这样修改之后,资本也就创造价值了)

提示:《资本论》第三卷第17章:“这些办事员的无酬劳动,虽然不会创造剩余价值,但会为他创造占有剩余价值的条件;这对这个资本来说,结果是完全一样的”。
\paragraph{回答}
这样的劳动者,并不属于剥削阶级。

要回答这个问题,需要考虑这些非生产劳动者,如企业管理人员,在剩余价值的生产中扮演的角色,马克思有过论述\footnote{\href{https://www.marxists.org/chinese/marx-engels/25/018.htm}{马克思:《资本论》第三卷第17章}}:\emph{“商人资本和剩余价值的关系不同于产业资本和剩余价值的关系。产业资本通过直接占有别人的无酬劳动来生产剩余价值。而商人资本使这个剩余价值的一部分从产业资本手里转移到自己手里,从而占有这部分剩余价值……商业工人不直接生产剩余价值。但是,他的劳动的价格是由他的劳动力的价值决定的,也就是由他的劳动力的生产费用决定的,而这个劳动力的应用,作为力的一种发挥,一种表现,一种消耗,却和任何别的雇佣工人的情况一样,是不受他的劳动力的价值的限制的。因此,他的工资和他帮助资本家实现的利润量之间,不保持任何必要的比例。资本家为他支出的费用,和他带给资本家的利益,是不同的量。他给资本家带来利益,不是因为他直接创造了剩余价值,而是因为他在完成一部分无酬劳动的时候,帮助资本家减少了实现剩余价值的费用。”}

因此,现代社会中第三产业的劳动者所扮演的角色,和马克思提到的商业工人的角色是一致的。他们并非同产业工人一样直接生产商品(创造价值),但是他们的劳动力被用于帮助资本家减少实现剩余价值的费用(即被用于实现价值):教师的劳动力被用于培养无产阶级;医生的劳动力被用于维修无产阶级;警察的劳动力被用于保护、看管无产阶级;销售人员的劳动力被用于推销商品……

从而,可以看到,从事非生产性劳动的劳动者在资本主义的生产关系中一样是被剥削的一方,会计等办事员经手可达上千万金额的账目,没有他们,资本的流转无从谈起,但这些人每月的工资数千而已,甚至由于这一行业的电算化,对于熟练白领的需求也越来越少,马克思如此描述:\emph{“资本主义生产方式越是使教学方法等等面向实践,随着科学和国民教育的进步,预备教育、商业知识和语言知识等等,就会越来越迅速地、容易地获得,越来越普及,越来越便宜地再生产出来。由于国民教育的普及,就可以从那些以前没有可能干这一行并且习惯于较差的生活方式的阶级中招收这种工人。这种普及增加了这种工人的供给,因而加强了竞争。”}这不就是当今工商管理等一众文科专业的学生的现状吗?

\paragraph{题目}
现在有的人声称“股份制改革就是马克思的重建劳动者个人所有制”。那么,让工人持有一部分股票,能不能起到重建劳动者个人所有制的作用?从而既保留生产资料的私人占有,又消除资本主义所有制的的弊端?

提示:《反杜林论》和《资本论》第三卷:
\emph{“工人自己的合作工厂,是在旧形式内对旧形式打开的第一个缺口,虽然它在自己的实际组织中,当然到处都再生产出并且必然会再生产出现存制度的一切缺点。但是,资本和劳动之间的对立在这种工厂内已经被扬弃,虽然起初只是在下述形式上被扬弃,即工人作为联合体是他们自己的资本家,也就是说,他们利用生产资料来使他们自己的劳动增殖。这种工厂表明,在物质生产力和与之相适应的社会生产形式的一定的发展阶段上,一种新的生产方式怎样会自然而然地从一种生产方式中发展并形成起来。没有从资本主义生产方式中产生的工厂制度,合作工厂就不可能发展起来;同样,没有从资本主义生产方式中产生的信用制度,合作工厂也不可能发展起来。信用制度是资本主义的私人企业逐渐转化为资本主义的股份公司的主要基础,同样,它又是按或大或小的国家规模逐渐扩大合作企业的手段。资本主义的股份企业,也和合作工厂一样,应当被看作是由资本主义生产方式转化为联合的生产方式的过渡形式,只不过在前者那里,对立是消极地扬弃的,而在后者那里,对立是积极地扬弃的。”}——《马克思恩格斯文集》第7卷第498页。

\paragraph{回答}
让工人持有一部分股票的股份制改革,就是让工人成为自己的资本家,重建劳动者对生产资料(为股票所代表)的占有权,将剩余价值收回到工人手中。然而,这并不能重建“劳动者个人所有制”。

首先,股份制改革的结果是工人联合起来作为资本家占有生产资料,或许可以称为“实验性劳动者联合所有制”。在社会化大生产的条件下,不可能再存在小农时代的劳动者个人便占有生产资料的情况了,依托股票信用制度,虽然将生产资料所有权分割到个人,但这是虚幻的个人所有制,当有人根据股票声索对应的生产资料时,只会发现强行分割生产资料意味着破坏生产资料,从而股份制改革也只带来了一种联合所有制。

此外,将股份制改革推向终点,就是所有股份均归工人所有、废除了雇主存在的合作社制度,工人们在合作社内齐心协力,自行生产。那么合作社制度怎么样呢?马克思曾对合作社大加称赞\footnote{\href{https://www.marxists.org/chinese/marx/marxist.org-chinese-marx-18640928.htm}{马克思:《国际工人协会成立宣言》}}:\emph{“但是,劳动的政治经济学对财产的政治经济学 还取得了一个更大的胜利,我们说的是合作运动,特别是由少数勇敢的“手”独力创办起来的合作工厂。对这些伟大的社会试验的意义不论给予多么高的估价都是不算过分的。工人们不是在口头上,而是用事实证明:大规模的生产,并且是按照现代科学要求进行的生产,在没有利用雇佣工人阶级劳动的雇主阶级参加的条件下是能够进行的;他们证明:为了有效地进行生产,劳动工具不应当被垄断起来作为统治和掠夺工人的工具;雇佣劳动,也像奴隶劳动和农奴劳动一样,只是一种暂时的和低级的形式,它注定要让位于带着兴奋愉快心情自愿进行的联合劳动。”}但是也应该看到,在资本主义社会中发展合作社制度,将会遇到行业托拉斯,垄断资本的阻挠,\emph{“不管合作劳动在原则上多么优越,在实际上多么有利,只要它仍然限于个别工人的偶然努力的狭隘范围,就始终既不能阻止垄断势力按照几何级数增长,也不能解放群众,甚至不能显著地减轻他们的贫困的重担……要解放劳动群众,合作劳动必须在全国范围内发展,因而也必须依靠全国的财力。但是土地巨头和资本巨头总是要利用他们的政治特权来维护和永久保持他们的经济垄断的。”}

那么,在全国范围内开展合作社,便必须实行社会主义革命,建立生产资料的公有制了。让工人持有股票的想法并不新鲜,已经证明这种改良的做法既不能实现劳动者对生产资料的私人占用,也不能消除资本主义所有制的弊端。
\end{document}