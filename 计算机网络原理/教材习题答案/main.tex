\documentclass[a4paper]{article}
\usepackage[left=1.5cm,right=1.5cm]{geometry}
\usepackage{ctex}
\usepackage{siunitx}
\usepackage{enumerate}
\usepackage{hyperref}
\usepackage{float}
\usepackage[thehwcnt=\hspace{-2pt}]{iidef}
\thecourseinstitute{清华大学}
\thecoursename{计算机网络原理}
\theterm{2023年秋季学期}
\hwname{计算机网络部分习题答案}
\begin{document}
\courseheader
\name{杨哲涵}
习题按照计算机网络(第6版,潘爱民译)编号.
\section{第一章}
\paragraph{1} 3种情况分别为
\begin{itemize}
    \item 数据速率为1099.511627776Mb/s
    \item 数据速率为733.0077518506666Mb/s
    \item 数据速率为549.755813888Mb/s
\end{itemize}
\paragraph{6}
跨国高速信道可能具有很高的带宽,但延迟较大.办公室内的一个局域网带宽较低,但延迟也较低.
\paragraph{8}
计算得刚好10颗卫星.
\paragraph{12}
5个路由器点到点连接共有10条路线,每条路线有4中线路选择,因此总共遍历时间为$4^{10}\times10\unit{\ms}=104857.6\unit{\s}$.
\paragraph{13}
可知通信过程关于根节点对称,因此每条消息经过跳数约为$2\Sigma_{i=1}^n2^{i-1}(i-1)/(2^n-1)=2(n\frac{2^n}{2^n-1}-2)$.在$n$很大的情况下为$2n-4$.
\paragraph{14}
信道被浪费的时间为总时间减去正常使用与无人使用的时间,即$1-n(1-p)^{n-1}p-(1-p)^n$
\paragraph{16}
每一层自己的功能需要若干信息及控制信号才能实现.此外,每一层不应修改将要递给上一层的数据.因此合理的做法是额外添加自己的源和目标等信息
\paragraph{17}按照从上到下的层次为:
\begin{itemize}
    \item 传输层: 字节流抽象
    \item 传输层: 可靠传递
    \item 传输层: 按序传递
    \item 网络层: 尽力传递
    \item 链路层: 点到点链路抽象
\end{itemize}
\paragraph{19}
65000B/s
\paragraph{20}
OSI要求在需要不同抽象的地方创建一层.公司总裁的抽象为提出合作意向.而讨论合作在经济方面的问题应当交给新的层,例如市场部门.
\paragraph{21}
不同,字节流是面向连接的,而消息流是无连接的.
\paragraph{24}
平均传输次数为$1/(1-p)$
\paragraph{25}
(a).OSI中的数据链路层,TCP/IP的链路层

(b).OSI中的网络层,TCP/IP中的互联网络层
\paragraph{32}
方案一的传输效率较高,因为只需要重传发生损坏的数据包,但存在因为发送方故障而接受损坏文件的可能.方案二相当低效,每次发生错误后需要重传整个文件.
\paragraph{38}
两个优点是,不同参与者不用担心兼容性问题;采用国际标准的人将受益于规模经济.两个缺点是,制定标准的过程常常有妥协,并不一定带来最好的技术;标准被广泛采用后难以更新替换.
\paragraph{41}
如果实现正确,不会影响.
\paragraph{42}
一定会影响第$k+1$层的服务,不会影响第$k-1$层的服务.
\paragraph{45}
北京-悉尼直线距离8760公里,延迟160ms,北京-剑桥(马萨诸塞州)直线距离10000公里,延迟37.9ms.可以看出地理距离只是影响延迟的诸多因素之一.
\section{第二章}
\paragraph{2}
衰减低,容量大,廉价,难以窃听.但是光纤在连接处的信号损失较大.
\paragraph{12}
无噪声情况下,最大数据速率为6000$\log_2V$b/s.如果信道上有噪声,且信噪比是30dB,那么最大数据速率为29901.68b/s.
\paragraph{13}
不依赖具体传输介质,因此适用.
\paragraph{14}
由奈奎斯特定理,速率为24Mb/s
\paragraph{38}
根据香农定理,至少需要2.82dB的信噪比.
\paragraph{47}
星状拓扑的最好,最坏,平均情形都为2跳.双向环的最好,最坏,平均情形为1跳,$n/2$跳,$n/4$跳.全联通的最好,最坏,平均情形都为1跳.
\paragraph{48}
电流交换的延迟为$s+x/b+kd$,数据包交换的延迟为$x/b+(k-1)p/b+kd$.可见,$sb>(k-1)p$时,即跳数小,网络传输速率大,数据包长度短的情况下数据包交换网络更优.
\paragraph{49}
注意这里要考虑包头长度造成总传输数据量的变化.共有$x/p$个包,需要$(p+h)x/(pb)$时间发出,存储转发引起的延迟为$(k-1)(p+h)/b$,总延迟为二者相加.因此$p=\sqrt{hx/(k-1)}$时,总延迟最小.
\paragraph{50}
可以发现,用三种颜色即可在相邻颜色不重复的情况下填充蜂窝.从而每个蜂窝网络可使用$840/3=280$个频率.
\paragraph{61}
在GEO高度(一个轨道面3颗),端到端传输时间为$(35800+6400)\times\frac{2\pi}{3}/300000=294.61$ms;MEO的情况(一个轨道面9颗)为64.93ms;LEO的情况(铱星,一个轨道面11颗卫星)为13.41ms.
\paragraph{62}
将经过5次中继,延迟为0.05ms,外加传输距离导致的延迟78.1998ms,总延迟为78.2498ms.
\section{第三章}
\paragraph{1}
不加处理的情况下会造成帧被错误地分割.因此通常的做法是用转义字符替代前导码.
\paragraph{2}
\verb|A B ESC ESC C ESC ESC ESC FLAG ESC FLAG D|
\paragraph{4}
\verb|0110 0111 1101 1110 11111 1|
\paragraph{5}
不是很理解这道题的意思,如果指的是Internet checksum.那么单个比特的错误有可能不被检测出来.例如:

\verb|0001|错误插入了一个\verb|0|比特,变成\verb|000010|.假设校验和计算方法为从头开始每4位作为一个整数采用一补数相加,那么最后加0不改变结果.这个错误没有被检测到.
\paragraph{6}
一次发送成功的概率为$0.8^{10}$.平均需要发送$1/0.8^{10}=9.31$次.
\paragraph{11}
数据为\verb|10101111|,偶校验海明编码后为\verb|hh1h010h1111|.可以参考\url{https://www.wikiwand.com/zh/Hamming_code}进行计算,结果为\verb|101001001111|
\paragraph{13}
收到\verb|101101001101|,可以发现没有错误.
\paragraph{24}
首先补齐3个\verb|0|,得到\verb|10011101000|,接着长除\verb|1001|,得到商为\verb|100|.结果为\verb|10011101000-100=10011101100|.
\paragraph{28}
设帧长为$l$,有$\frac{l/4}{l/4+20+20}>0.5$,可得$l>320$.因此需要帧长超过320比特.
\paragraph{29}
B的效率应当为A的20分之一,即5\%.
\paragraph{33}
对于旧信道的$BD$,我们知道$2BD+1=3$,有$BD=1$.升级为新信道后,$BD$变为2,$w$变为5,如果维持原先的协议不变,带宽效率为$2/5=60\%$.
\paragraph{35}
可算得$BD=54.28125$,有$w=109.5625<2^7$,取序号长度为7位比较合理.
\paragraph{42}
每次发送后会触发一次超时重传,平均传输2次.
\paragraph{44}这种情况下$BD=270$,$w=2BD+1=541$.
a. 信道利用率为1/541=0.185\%.

b. 利用率为8/541=1.48\%.

c. 利用率为8/541=0.739\%.
\section{第四章}
\paragraph{6} (a).CSMA/CD可被看作分槽ALOHA.时间槽长度为$2\tau=4/(3\times10^5\times0.82)=0.0162$ms.

(b).时间槽长度为$2\tau=80/(3\times10^5\times0.65)=0.4103$ms.
\paragraph{10}
他们的使用场景为CPU与网络敏感的视频游戏,IO并不重要.令牌环协议以确定次序分配带宽,延迟相比非坚持CSMA更低,此外,非坚持CSMA可能因为随机等待重发引入抖动.采用令牌环协议更稳定,更适合这种场景.
\paragraph{14}
(a).由于任意两个节点间都可以进行对角线铺设,所以铺设长度为$4\sum_{i=1}^7\sum_{j=1}^{15}\sqrt{(i-4)^2+(j-8)^2}=1831.77$米

(b).使用经典以太网IEEE 802.3 LAN,可以看作一根线缆不重不漏地连接了所有主机,需要$14\times4\times7+6\times4=416$米线缆.
\paragraph{15}
由于经典10Mbps以太网使用Machester编码,所以波特率为比特率的两倍,即20Mbaud.
\paragraph{16}
Machester编码中,\verb|1|被编码为低电平到高电平的跳变,\verb|0|被编码为高电平到低电平的跳变.因此比特流\verb|0001110101|将被编码为\verb|-_-_-__-_-_--__--__-|.
\paragraph{17}
由于已经假定了没有冲突,此外由于确认帧的存在,发送方不需要使发送时间达到二倍信道传输延迟来检测冲突.因此传输数据帧以及接受确认帧总共花费的时间为$256/10M+1k/200\mu+32/10M+1k/200M$,除去开销的有效数据速率为$(256-32)/t=5.773$Mpbs.
\paragraph{18}
对于题目的情况,$2\tau=10\mu s$.需要帧长至少达到可以发送$2\tau$的长度.因此最小帧长为$1G\times10\mu=10000$bit.
\paragraph{20}
这个问题可以通过限制快速以太网的线路长度为原先的1/10来解决.如此可以维持最小帧长不变.
\paragraph{41}
每次操作后$B_2$的哈希表列在下方:

(a). \verb|B->4|

(b). \verb|B->4;F->2|

(c). \verb|B->4;F->2|

(d). \verb|B->4;F->2;G->3|

(e). \verb|B->4;F->2;G->3;D->1|

(f). \verb|B->4;F->2;G->3;D->1|
\paragraph{42}
类似上述过程.

(a) 会导致$B_1$与$B_2$均进行广播.之后哈希表分别为\verb|A->1|以及\verb|A->4|

(b) 会导致$B_1$与$B_2$均进行广播.之后哈希表分别为\verb|A->1;B->2|以及\verb|A->4;B->4|

(c) 不会广播.之后哈希表分别为\verb|A->1;B->2;c->3|以及\verb|A->4;B->4|

(d) $B_2$会广播.之后哈希表分别为\verb|A->1;B->2;C->3;G->4|以及\verb|A->4;B->4;G->3|

(e) 会导致$B_1$与$B_2$均进行广播.之后哈希表分别为\verb|A->1;B->2;C->3;G->4;E->4|以及\verb|A->4;B->4;G->3;E->2|

(f) $B_2$会广播.之后哈希表分别为\verb|A->1;B->2;C->3;G->4;E->4;D->4|以及\verb|A->4;B->4;G->3;E->2;D->1|
\paragraph{43}
不是很清楚这道题的意思,按照网桥.端口的格式列出为:

(a) \verb|B1.{2,3,4}, B2.{1,2,3}|

(b) \verb|B1.{1,2,3}, B2.{1,3,4}|

(c) 此时在$B_2$处就会将帧丢弃(因为端口相同),不会有转发

(d) \verb|B2.2|

(e) \verb|B1.1, B2.4|

(f) \verb|B1.{1,3,4}, B2.2|
\paragraph{44}
加入$B_0$后,由于其具有最低标识符,$B_0$应当是新的根.另外考虑同样跳数下的选择较低标识符,以使路径唯一.可构造新的生成树如下:\verb|B0(-B5)-B4(-B3)-B2-B1|.其中$B_5$与$B_0$相连,$B_4$与$B_3$相连.
\paragraph{47}
这两个网络都需要CSMA/CD.考虑网络A,尽管工作方式为全双工,但由于通过集线器连接,集线器会向所有其他线路转发帧,同时到达时有可能发生冲突.对于网络B,由于站是通过半双工线缆连接到交换机的,不能同时发送和接受,因此也需要CSMA/CD.
\section{第五章}
\paragraph{4}
根据链路状态,可以构建全局的图,计算最短路径得到\verb|B|的新的路由表如下:
\begin{table}[H]
    \begin{tabular}{|llll|}
        \hline
        \verb|des| & \verb|nexthop| & \verb|routes|     & \verb|cost| \\
        \verb|A|   & \verb|C|       & \verb|D->C->E->A| & \verb|8|    \\
        \verb|B|   & \verb|F|       & \verb|D->F->B|    & \verb|5|    \\
        \verb|C|   & \verb|C|       & \verb|D->C|       & \verb|3|    \\
        \verb|D|   & \verb|D|       & \verb|D|          & \verb|0|    \\
        \verb|E|   & \verb|C|       & \verb|D->C->E|    & \verb|6|    \\
        \verb|F|   & \verb|F|       & \verb|D->F|       & \verb|4|    \\ \hline
    \end{tabular}
\end{table}
\paragraph{7}
根据距离向量,\verb|C|需要为每条可能的线路计算点到点开销与预报的开销之和,得到路由表如下:
\begin{table}[H]
    \begin{tabular}{|llll|}
        \hline
        \verb|des| & \verb|nexthop| & \verb|routes|  & \verb|cost| \\
        \verb|A|   & \verb|B|       & \verb|C->B->A| & \verb|11|   \\
        \verb|B|   & \verb|B|       & \verb|C->B|    & \verb|6|    \\
        \verb|C|   & \verb|C|       & \verb|C|       & \verb|0|    \\
        \verb|D|   & \verb|D|       & \verb|C->D|    & \verb|3|    \\
        \verb|E|   & \verb|E|       & \verb|C->E|    & \verb|5|    \\
        \verb|F|   & \verb|B|       & \verb|C->B->F| & \verb|8|    \\ \hline
    \end{tabular}
\end{table}
\paragraph{8}
交换指的是交换机在数据链路层将帧从一个端口转发到另一个端口.转发指的是路由器在网络层查询路由表,将数据包发送到下一跳.路由指的就是路由器在网络层不断通过链路状态或距离向量学习拓扑,选择最短路径,维护更新路由表的过程.
\paragraph{16}
(a). 对于每次传输,$P(1)=p,P(2)=(1-p)p,P(3)=(1-p)^2$, 因此每次传输的平均跳数为$3-3p+p^2$.

(b). 数据包在传输时经过3跳才算传输成功,因此数据包的平均传输次数为$\sum_{n=1}^{+\infty}n(1-(1-p)^2)^{n-1}(1-p)^2=\frac{1}{(1-p)^2}$.另一种想法为,每次传输成功概率为$(1-p)^2$,因此平均需要传输$1/(1-p)^2$次.

(c). 对于每个收到的数据包,平均跳数为$(3-3p+p^2)\frac{1}{(1-p)^2}=\frac{3-3p+p^2}{(1-p)^2}$.
\paragraph{21}
这里的300MB指的是令牌的量,首先主机可以全速发送30s共450MB数据,此后550MB都需要按照5MBps的速率发送,另外需要110s.因此发送1000MB总共需要140s.
\paragraph{23}
可知$10t+1=50t$,从而一个突发流量最大持续时间为$t=1/40=25$ms.
\paragraph{27}
由于段偏移字段以8字节为单位,因此将IP数据包分片后,每个分片的长度(除去包头)都应当是8的倍数.
\begin{table}[H]
    \begin{tabular}{|llllll|}
        \hline
                              & \verb|总长度| & \verb|标识符| & \verb|DF| & \verb|MF| & \verb|段偏移| \\
        \verb|A-R1  MTU=1010| & \verb|940| & \verb|0|   & \verb|0|  & \verb|0|  & \verb|0|   \\
        \verb|R1-R2 MTU=504|  & \verb|500| & \verb|0|   & \verb|0|  & \verb|1|  & \verb|0|   \\
                              & \verb|460| & \verb|0|   & \verb|0|  & \verb|0|  & \verb|60|  \\
        \verb|R2-B  MTU=500|  & \verb|500| & \verb|0|   & \verb|0|  & \verb|1|  & \verb|0|   \\
                              & \verb|460| & \verb|0|   & \verb|0|  & \verb|0|  & \verb|60|  \\ \hline
    \end{tabular}
\end{table}
\paragraph{33}
\verb|255.255.240.0|代表的掩码长度为20,因此有$2^{32-20}=2^{12}=4096$个地址.但全0和全1的地址都不可用,因此有4094个可以分配给主机的地址.
\paragraph{35}
对于A组织申请的4000个地址,\verb|198.16.0.0/20|是分配出的第一个地址,\verb|198.16.15.255/20|是分配出的最后一个地址.

对于B组织的2000个地址,\verb|198.16.16.0/21|是第一个地址,\verb|198.16.23.255/21|是最后一个地址.

对于C组织的4000个地址,\verb|198.16.32.0/20|是第一个地址,\verb|198.16.47.255/20|是最后一个地址.

对于D组织的8000个地址,\verb|198.16.64.0/19|是第一个地址,\verb|198.16.127.255/19|是最后一个地址.
\paragraph{36}
可以被聚合,它们都可以被聚合到\verb|57.6.96.0/19|中.
\paragraph{37}
习题印刷错误,\verb|128|应为\verb|127|.没有必要,因为路由的时候是最长前缀匹配,所以只需要在路由表中添加一个新的项代表这1024个地址即可.新的地址块为\verb|29.18.60.0/22|.
\paragraph{55}
我使用\verb|traceroute -q 1|命令测试了题目中的一部分域名,并且参考了\href{https://www.gd1214b.icu/post/qfDddx_kZ/}{国内至国际骨干 ISP 线路整理},得到的结果如下:
\begin{table}[H]
    \begin{tabular}{|lll|}
        \hline
        \verb|域名|                & \verb|跨国链路|                                                            & \verb|有关的ISP|        \\
        \verb|www.u-tokyo.ac.jp| & \verb|219.158.9.217(中国联通,北京)->193.251.250.5(France,Paris)|             & \verb|联通AS4837|      \\
                                 & \verb|193.251.248.124(France)->58.138.83.186(USA,Los Angeles)|         &                      \\
                                 & \verb|58.138.83.186(USA,Los Angeles)->58.138.88.113(Japan,Osaka)|      &                      \\
        \verb|www.usyd.edu.au|   & \verb|101.4.114.222(CERNET,广东)->203.131.254.213(Hong Kong)|            & \verb|CERNET AS4538| \\
                                 & \verb|203.131.254.213(Hong Kong)->129.250.7.66(Singapore)|             &                      \\
                                 & \verb|15.230.212.85(Amazon,Singapore)->129.250.7.66(Amazon,Australia)| &                      \\
        \hline
    \end{tabular}
\end{table}
\section{第六章}
\paragraph{8}
数据链路层滑动窗口协议的超时时间是固定的,传输层的超时时间可能是动态的(RTT).此外数据链路层的超时时间一般为毫秒或微秒级别,而传输层的超时时间一般为毫秒或秒级别.
\paragraph{13}
考虑某个起始不公平的分配,乘法递增加法递减策略会使操作点向远离原点方向移动,随后沿着\ang{45}方向靠近原点,这一过程中操作点远离了公平线,最终路径会收敛到远离公平线的位置.对于加法递增加法递减策略与乘法递增乘法递减策略,它们既不会远离公平线,也不会靠近公平线,而是沿着路径振荡.因此三种策略都不能达到公平的目的.
\paragraph{14}
会收敛,因为采用平方根递减策略后,分配到较大带宽的用户减小的带宽也较大,这个算法会收敛到公平的分配.
\paragraph{15}
主机A获得更高的吞吐量,由于主机A速率为800kbps,主机B为600kbps,两者总速率超过1Mbps.因此会遭遇拥塞控制,即TCP传输的速率会降低,UDP的速率不变.最终主机A获得800kbps,主机B获得200kbps.
\paragraph{16}
直接发送原始的IP数据包,意味着用户进程需要处理裸露的网络层的数据,可想而知,一个IP数据包归哪个进程所有将是一个问题.如果有人决定根据IP数据包内的特定字段决定进程归属,相当于变相的UDP协议.这样人们应当相信,无论如何,为了提供进程级别的网络接口,需要某个统一的协议将IP数据包进行分发.
\paragraph{20}
考虑不同操作系统进行网络通信,它们对于进程的概念可能并不统一,难以对进程ID达成共识.此外借助端口这个概念,人们可以保留一些知名端口(如\verb|80, 443, 22|)用于提供服务端的监听,但要想借助进程ID,就很难做到这一点.
\paragraph{26}
有可能,例如主机1有多个IP地址,例如一个IPv4地址,一个IPv6地址.总体上讲,如果主机1有$m$个IP地址,主机2有$n$个,那么在这一条件下这两个端口之间可以有$m\times n$个TCP连接.
\paragraph{32}
第一个10ms后传输2KB数据,第二个10ms后传输4KB数据,第三个10ms后传输8KB数据,第四个10ms后传输16KB数据,此时拥塞窗口已经变为32KB>24KB.即经过40ms后才能让第一个满窗口的数据被发送出去.
\paragraph{33}
超时后,慢启动重新开始,阈值设为一半,为9KB,拥塞窗口从1KB开始.前四次突发传输分别发送1KB,2KB,4KB,8KB.由于不能超过阈值,这之后拥塞窗口为8KB.
\paragraph{36}
双向延迟为20ms,由于窗口已满,每发65535个字节都需要等待确认,在1Gbps线路上发65535B需要0.52428ms,外加延迟20ms.因此可以达到的最大吞吐量为25.544Mbps,信道利用率为2.554\%.
\end{document}
