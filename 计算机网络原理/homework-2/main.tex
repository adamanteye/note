\documentclass[a4paper]{article}
\usepackage[left=1.5cm,right=1.5cm]{geometry}
\usepackage{ctex}
\usepackage{siunitx}
\usepackage{enumerate}
\usepackage[thehwcnt = 2]{iidef}
\thecourseinstitute{清华大学}
\thecoursename{计算机网络原理}
\theterm{2023年秋季学期}
\hwname{作业}
\begin{document}
\courseheader
\name{杨哲涵}
习题按照计算机网络(第6版,潘爱民译)编号.
\section{第一章}
\paragraph{1} 3种情况分别为
\begin{itemize}
    \item 数据速率为1099.511627776Mb/s
    \item 数据速率为733.0077518506666Mb/s
    \item 数据速率为549.755813888Mb/s
\end{itemize}
\paragraph{6}
跨国高速信道可能具有很高的带宽,但延迟较大.办公室内的一个局域网带宽较低,但延迟也较低.
\paragraph{8}
计算得刚好10颗卫星.
\paragraph{12}
5个路由器点到点连接共有10条路线,每条路线有4中线路选择,因此总共遍历时间为$4^{10}\times10\unit{\ms}=104857.6\unit{\s}$.
\paragraph{13}
可知通信过程关于根节点对称,因此每条消息经过跳数约为$2\Sigma_{i=1}^n2^{i-1}(i-1)/(2^n-1)=2(n\frac{2^n}{2^n-1}-2)$.在$n$很大的情况下为$2n-4$.
\paragraph{14}
信道被浪费的时间为总时间减去正常使用与无人使用的时间,即$1-n(1-p)^{n-1}p-(1-p)^n$
\paragraph{16}
每一层自己的功能需要若干信息及控制信号才能实现.此外,每一层不应修改将要递给上一层的数据.因此合理的做法是额外添加自己的源和目标等信息
\paragraph{17}按照从上到下的层次为:
\begin{itemize}
    \item 传输层: 字节流抽象
    \item 传输层: 可靠传递
    \item 传输层: 按序传递
    \item 网络层: 尽力传递
    \item 链路层: 点到点链路抽象
\end{itemize}
\paragraph{19}
65000B/s
\paragraph{20}
OSI要求在需要不同抽象的地方创建一层.公司总裁的抽象为提出合作意向.而讨论合作在经济方面的问题应当交给新的层,例如市场部门.
\paragraph{21}
不同,字节流是面向连接的,而消息流是无连接的.
\paragraph{24}
平均传输次数为$1/(1-p)$
\paragraph{25}
(a).OSI中的数据链路层,TCP/IP的链路层

(b).OSI中的网络层,TCP/IP中的互联网络层
\paragraph{32}
方案一的传输效率较高,因为只需要重传发生损坏的数据包,但存在因为发送方故障而接受损坏文件的可能.方案二相当低效,每次发生错误后需要重传整个文件.
\paragraph{38}
两个优点是,不同参与者不用担心兼容性问题;采用国际标准的人将受益于规模经济.两个缺点是,制定标准的过程常常有妥协,并不一定带来最好的技术;标准被广泛采用后难以更新替换.
\paragraph{41}
如果实现正确,不会影响.
\paragraph{42}
一定会影响第$k+1$层的服务,不会影响第$k-1$层的服务.
\paragraph{45}
北京-悉尼直线距离8760公里,延迟160ms,北京-剑桥(马萨诸塞州)直线距离10000公里,延迟37.9ms.可以看出地理距离只是影响延迟的诸多因素之一.
\section{第二章}
\paragraph{2}
衰减低,容量大,廉价,难以窃听.但是光纤在连接处的信号损失较大.
\paragraph{12}
无噪声情况下,最大数据速率为6000$\log_2V$b/s.如果信道上有噪声,且信噪比是30dB,那么最大数据速率为29901.68b/s.
\paragraph{13}
不依赖具体传输介质,因此适用.
\paragraph{14}
由奈奎斯特定理,速率为24Mb/s
\paragraph{38}
根据香农定理,至少需要2.82dB的信噪比.
\paragraph{47}
星状拓扑的最好,最坏,平均情形都为2跳.双向环的最好,最坏,平均情形为1跳,$n/2$跳,$n/4$跳.全联通的最好,最坏,平均情形都为1跳.
\paragraph{48}
电流交换的延迟为$s+x/b+kd$,数据包交换的延迟为$x/b+(k-1)p/b+kd$.可见,$sb>(k-1)p$时,即跳数小,网络传输速率大,数据包长度短的情况下数据包交换网络更优.
\paragraph{49}
注意这里要考虑包头长度造成总传输数据量的变化.共有$x/p$个包,需要$(p+h)x/(pb)$时间发出,存储转发引起的延迟为$(k-1)(p+h)/b$,总延迟为二者相加.因此$p=\sqrt{hx/(k-1)}$时,总延迟最小.
\paragraph{50}
可以发现,用三种颜色即可在相邻颜色不重复的情况下填充蜂窝.从而每个蜂窝网络可使用$840/3=280$个频率.
\paragraph{61}
在GEO高度(一个轨道面3颗),端到端传输时间为$(35800+6400)\times\frac{2\pi}{3}/300000=294.61$ms;MEO的情况(一个轨道面9颗)为64.93ms;LEO的情况(铱星,一个轨道面11颗卫星)为13.41ms.
\paragraph{62}
将经过5次中继,延迟为0.05ms,外加传输距离导致的延迟78.1998ms,总延迟为78.2498ms.
\end{document}
